\documentclass[final]{siamart171218}
\usepackage{amsmath}
\usepackage{amssymb}
\usepackage{bm}

\setlength{\oddsidemargin}{0.65in}
\setlength{\evensidemargin}{0.65in}

\title{Differential count analysis with a topic model}

\author{Peter Carbonetto\thanks{Dept. of Human Genetics and the Research Computing Center, University of Chicago, Chicago, IL}}

\begin{document}

\maketitle

\section{Motivation, and overview of methods}

The aim of this document is to derive, from first principles, a method
for analysis of differential gene expression using a topic model (also
known as ``grade of membership model'' \cite{dey-2017}). This method
may have other uses --- say, to identify ``key words'' in a topic
modeling analysis of text documents --- but since our main motivation
is analysis of gene expression data, we describe the methods with that
application in mind.

For motivation, we begin with the ``log-fold change'' statistic
commonly used in microarray and RNA sequencing experiments to quantify
expression differences between two conditions (e.g.,
\cite{cui-churchill-2003, quackenbush-2002}). The log-fold change for
gene $j$ and condition $k$ is a ratio of two conditional expectations,
\begin{equation}
\beta_{jk} \equiv
\log_2 \frac{E[\,x_j \,|\, \mathrm{condition} = k\,]}
            {E[\,x_j \,|\, \mathrm{condition} \neq k\,]},
\label{eq:lfc}
\end{equation}
where $x_j$ is the measured expression level of gene
$j$.\footnote{Defining $x_{jk}$ as the total gene expression for gene
  $j$ among all samples (expression profiles) in condition $k$, $x_j$
  as the total gene expression for gene $j$ in all samples, $n_k$ as
  the number of samples in condition $k$, and $n$ as the total sample
  size, the log-fold change can be computed as $\beta_{jk} = \log_2
  \Big\{ \frac{x_{jk}}{x_j - x_{jk}} \times \frac{n - n_k}{n_k}
  \Big\}$.} The way in which the expression level $x_j$ is defined can
lead to different log-fold change statistics. For example, several
popular methods for analyzing differential expression compare changes
in {\em relative} expression (say, relative to total expression in a
cell) \cite{bullard-2010, voom, DESeq2, edgeR}. As we will see, a
topic modeling perspective provides a natural way to analyze
differential gene expression when comparing either relative or
absolute expression levels.

Topic modeling brings a new twist to analysis of differential gene
expression: In conventional differential expression analysis, each
sample is assigned to a single condition; in topic modeling, the
assignments to each topic are {\em proportional.} This needs to be
accounted for in developing the new methods for differential
expression analysis.

\subsection{The binomial model}

We begin with a simple binomial model that predicts expression of a
single gene given the proportional assignments to a topic:
\begin{equation}
x_i \sim \mathrm{Binom}(s_i, \pi_i).
\label{eq:binomial}
\end{equation}
Here, $x_i$ is the expression level of the target gene in sample $i$,
$s_i$ is the total expression in sample $i$, and $\mathrm{Binom}(n,
\theta)$ denotes the binomial distribution with $n$ trials and success
probability $\theta$. In this simple model, the binomial probabilities
are defined as
\begin{equation}
\pi_i = (1 - q_i) p_0 + q_i p_1,
\label{eq:binomial-prob}
\end{equation}
where $q_i \in [0,1]$ is the (known) proportion of sample $i$ that is
attributed to the topic, and $p_0, p_1 \in [0, 1]$ are two unknowns to
be estimated.\footnote{This is a special case of the {\em linear
    probability model} in econometrics called \cite{stock-watson}. The
  linear probability model would be typically written as $\pi_i = b_0
  + q_i b$, where $b_0 = p_0$ and $n = p_1 - p_0 \in [-1, +1]$. This
  is a regression model for the binomial probability $\pi_i$, in which
  $\pi_i$ increases linearly with topic proportion $q_i$.}

Statistical inference with this simple binomial model implements
analysis of differential gene expression. In particular,
$\beta^{\mathsf{rel}} \equiv \log_2(p_1/p_0)$ is the {\em log-fold
  change in relative expression.} To show that this is so, consider
the following statistical process for generating the counts $x_1,
\ldots, x_n$:
\vspace{1em}
\begin{itemize}

\item for $i = 1, \ldots, n$
\begin{enumerate}

  \item for $t = 1, \ldots, s_i$
  \begin{enumerate}

  \item Sample a topic, $z_{it} \sim \mathrm{Binom}(1,q_i)$.

  \item Sample a gene, $w_{it} \,|\, z_{it} \sim \left\{\begin{array}{ll}
  \mathrm{Binom}(1,p_1) & \mbox{if $z_{it} = 1$} \\
  \mathrm{Binom}(1,p_0) & \mbox{otherwise.}
  \end{array}\right.$

  \end{enumerate}

  \item Generate the final gene count, 
  $x_i \leftarrow w_{i1} + \cdots + w_{is_i}$.

\end{enumerate}
\end{itemize}
\vspace{1em} In this statistical process, $q_i = p(z_{it} = 1)$ is the
topic probability, $p_1 = p(w_{it} = 1 \,|\, z_{it} = 1)$ is the
conditional probability that the gene is expressed given membership to
the topic, and $p_0 = p(w_{it} = 1 \,|\, z_{it} = 0)$ is the
probability that the gene is expressed when not belonging to the
topic. Therefore, we have
\begin{equation}
\beta^{\mathsf{rel}} \equiv \log_2 \frac{p_1}{p_0}  
= \log_2 \frac{p(w_{it} = 1 \,|\, z_{it} = 1)}
            {p(w_{it} = 1 \,|\, z_{it} = 0)}.
\label{eq:lfc-binom}
\end{equation}
{\em This is the log-fold change statistic for relative expression
  given proportional topic assignments $q_1, \ldots, q_n$.} The
binomial model \eqref{eq:binomial} can in fact be derived from this
statistical process (proof not shown). Therefore, estimating $p_0,
p_1$ for the binomial model \eqref{eq:binomial} provides an estimate
of the log-fold change \eqref{eq:lfc-binom}.

Before continuing, we point out that the $s_i$'s, in practice, do not
need to be the total expression for each sample $i$ ({\em i.e.}, the
total counts); for example, some researchers have suggested setting
$s_i$ to some pre-defined quantile of the sample's nonzero count
distribution (e.g., \cite{bullard-2010}). This approach has been
motivated in settings where a small number of genes are much more
highly expressed than the others, and therefore these few genes have a
large effect relative expression levels. In short, the binomial model
is quite flexible, and can accommodate different relative differential
expression analyses defines the $s_i$'s. The only constraint on $s_i$
is that it cannot be smaller than $x_i$.

\subsection{The Poisson model}

A Poisson model similar to the binomial model \eqref{eq:binomial}
leads to a method for estimating log-fold change in absolute
expression levels. We proceed in a similar way. The Poisson model
predicts expression $x_i$ given the topic proportions $q_i$:
\begin{equation}
x_i \sim \mathrm{Poisson}(t_i \lambda_i),
\label{eq:poisson}
\end{equation}
in which the Poisson rates are defined as
\begin{equation}
\lambda_i = (1 - q_i) f_0 + q_i f_1,
\end{equation}
and the two unknowns to be estimated are $f_0, f_1 \geq 0$. (The
scalars $t_i > 0$ are additional ``size factors'' that give the model
more flexibility; for now, assume $t_i = 1$ for all $i = 1, \ldots,
n$.) Observe that, unlike the binomial model, the unknowns for the
Poisson model are not constrained to be in $[0, 1]$, and so they
cannot represent probabilities. In the following, we show that
$\beta^{\mathsf{abs}} \equiv \log_2(f_1/f_0)$ is the log-fold change
in absolute expression.

Consider the following process for generating the counts $x_1, \ldots,
x_n$:
\begin{itemize}

\item for $i = 1, \ldots, n$
\begin{enumerate}

\item $a_i \sim \mathrm{Poisson}(f_1)$ 
\hfill Sample the within-topic gene count. \hspace{2em}

\item $b_i \sim \mathrm{Poisson}(f_0)$ 
\hfill Sample the outside-topic gene count. \hspace{2em}

\item $a_i' \sim \mathrm{Binom}(a_i, q_i)$ 
\hfill Subsample the within-topic genes. \hspace{2em}

\item $b_i' \sim \mathrm{Binom}(b_i, 1-q_i)$ 
\hfill Subsample the outside-topic genes. \hspace{2em}

\item $x_i \leftarrow a_i' + b_i'$ 
\hfill Generate the final gene count. \hspace{2em}

\end{enumerate}
\end{itemize}
In this generative process, $f_1 = E[a_i]$ represents the
within-topic gene rate, and $f_0 = E[b_i]$ is the outside-topic gene
rate, and therefore 
\begin{equation}
\beta^{\mathsf{abs}} \equiv 
\log_2 \frac{f_1}{f_0} = \log_2 \frac{E[a_i]}{E[b_i]}
\label{eq:lfc-poisson}
\end{equation}
is the {\em log-fold change statistic in absolute expression given
  proportional topic assignments $q_1, \ldots, q_n$.} For intuition,
when all the topic proportions $q_i$ are 0 or 1, this statistical
process simplify to $x_i \sim \mathrm{Poisson}(f_1)$ if $q_i = 1$, and
$x_i \sim \mathrm{Poisson}(f_0)$ if $q_i = 0$, and so
\eqref{eq:lfc-poisson} would reduce to the ratio of the mean
expression levels inside and outside the topic. The Poisson model
\eqref{eq:poisson} can be derived from this statistical process, and
so estimating $f_0, f_1$ for the Poisson model \eqref{eq:poisson}
provides an estimate of the log-fold change \eqref{eq:lfc-poisson}.

A practical question is the choice of size factors $t_i$. In the
standard analysis, one would set $t_i = 1$ for all $i = 1, \ldots, n$,
but there may be situations in which other settings for $t_i$ are
warranted. For example, you may know in advance that one one of the
samples, say, the first sample, $i = 1$, due to some technical error,
produced twice as much gene expression (say, two cells were
accidentally combined into one during sample preparation, and so
expression was measured in both). With $t_1 = 1$, this sample would
bias the log-fold change statistics, whereas setting $t_1 = 2$ would
``correct'' this bias.

\subsection{Binomial vs. Poisson: which to use}

The binomial model \eqref{eq:binomial} would be used to implement the
log-fold change analysis for relative expression, and the Poisson
model \eqref{eq:poisson} would implement the log-fold change analysis
for absolute expression. There is, as we have already established in
related writeups, a close relationship between the Poisson and
binomial models (the binomial being a special case of the
multinomial), and in fact we have exploited that close relationship to
use fast non-negative matrix algorithms to fit the multinomial topic
model. In particular, it can be shown that the likelihood for the
binomial model \eqref{eq:binomial} is the same as the likelihood for
this Poisson model:
\begin{align*}
x_i &\sim \mathrm{Poisson}(s_i \pi_i) \\
y_i &\sim \mathrm{Poisson}(s_i(1-\pi_i)),
\end{align*}
where $y_i = s_i - x_i$. So the key difference between the relative
and absolute expression analyses is actually not the use of the
Poisson or binomial, {\em but the way in which the models are
  parameterized.} This difference in parameterization leads to
analysis of absolute or relative expression. Therefore, we prefer to
keep the development of the Poisson and binomial models separate.

The next sections derive the mathematical expressions needed to
implement differential count analysis based on the binomial and
Poisson models.

\section{Binomial model derivations}

{\em Add derivations here.}

\section{Poisson model derivations}

To derive an algorithm for computing MLEs of $f_0, f_1$ in the Poisson
model \eqref{eq:poisson}, we begin with the log-likelihood,
\begin{equation}
\ell(f_0, f_1) \equiv \log p(x \,|\, f_0, f_1) = 
\sum_{i=1}^n x_i \log (t_i \lambda_i) - t_i \lambda_i +
\mbox{const,}
\label{eq:loglik-poisson}
\end{equation}
in which the ``const'' captures all terms that do not depend on $f_0$
or $f_1$. A useful identity is that the partial derivative of the
log-likelihood with respect to $\lambda_i$ is
\begin{equation*}
\frac{\partial\ell}{\partial\lambda_i} = \frac{x_i}{\lambda_i} - t_i.
\end{equation*}
Making use of this result, the partial derivatives of the
log-likelihood with respect to the model parameters $f_0$ and $f_1$
are
\begin{align}
\frac{\partial\ell}{f_0} &= 
\sum_{i=1}^n (x_i/\lambda_i - t_i) \times (1-q_i) \\
\frac{\partial\ell}{f_1} &= 
\sum_{i=1}^n (x_i/\lambda_i - t_i) \times q_i.
\end{align}
We use a quasi-Newton method implemented in the R function {\tt
  optim} to minimize the negative log-likelihood
\eqref{eq:loglik-poisson}.

\subsection{EM for Poisson model} 

We also implement a simple EM algorithm for fitting the Poison model
\eqref{eq:poisson}. The key is to introduce latent variables $a_i \sim
\mathrm{Poisson}(t_i (1-q_i) f_0)$ and $b_i \sim \mathrm{Poisson}(t_i
q_i f_1)$, then work with the expected complete log-likelihood
$E[\,\log p(x, a, b \,|\, f_0, f_1)\,]$. The ``M-step'' updates work
out to
\begin{align}
f_0 &= \frac{\sum_{i=1}^n \phi_i}{\sum_{i=1}^n t_i(1-q_i)} \\
f_1 &= \frac{\sum_{i=1}^n \gamma_i}{\sum_{i=1}^n t_i q_i},
\end{align}
where we have introduced notation for the posterior expectations of
the latent variables, $\phi_i \equiv E[a_i]$ and $\gamma_i \equiv
E[b_i]$. It can be shown that the posterior distribution of $(a_i,
b_i)$ is multinomial with number of trials $x_i$ and event
probabilities proportional to $(1-q_i) f_0$ and $q_i f_1$. So the
posterior expectations computed in the ``E-step'' are
\begin{align}
\phi_i   &= x_i (1 - q_i) f_0 / \lambda_i \\
\gamma_i &= x_i q_i f_1 / \lambda_i.
\end{align}
This completes the description of the EM algorithm for the Poisson
model.

\subsection{glm identity parameterization}

Once we have obtained MLEs of $f_0$ and $f_1$, we would also like to
characterize uncertainty in these estimates---that is, compute the
standard error (s.e.). In particular, we are interested in the s.e. of
the log-fold change statistic, $\beta^{\mathsf{abs}}$. As an
intermediate step, we consider the ``glm identity'' parameterization
$\lambda_i = b_0 + q_i b$, where $b = f_1 - f_0$ and $b_0 = f_0$. This
parameterization can be implemented using {\tt glm} in R with {\tt
  family = poisson(link = "identity")}, and therefore can be used to
verify our calculations.

Under the Laplace approximation to the likelihood at the MLE
$(\hat{b}_0, \hat{b})$, the covariance matrix is $-H^{-1}$, where $H$
is the $2 \times 2$ matrix of second-order partial derivatives,
\begin{equation*}
\frac{\partial^2\ell}{\partial b_0^2} = 
-\sum_{i=1}^n \frac{x_i}{\lambda_i^2}, \qquad
\frac{\partial^2\ell}{\partial b^2} = 
-\sum_{i=1}^n \frac{x_i q_i^2}{\lambda_i^2}, \qquad
\frac{\partial^2\ell}{\partial b_0 \partial b} = 
-\sum_{i=1}^n \frac{x_i q_i}{\lambda_i^2}.
\end{equation*}
This result can be used to obtain the standard errors and $z$-scores
for $\hat{b}_0$ and $\hat{b}$.

\subsection{Log-fold change parameterization}

Here we derive the s.e. for the MLE of $\beta \equiv \log (f_1/f_0)$. 
With this new parameterization, the Poisson rates are
\begin{equation}
\lambda_i = f_0 \times \{1 - q_i(1-e^{\beta})\}
\end{equation}
The second-order partial derivatives needed to compute the $2 \times
2$ Hessian are
\begin{align}
\frac{\partial\ell}{f_0} &= \frac{1}{f_0} \sum_{i=1}^n 
x_i - t_i\lambda_i \\
\frac{\partial\ell}{\beta} &= f_1 \sum_{i=1}^n 
(x_i/\lambda_i - t_i) \times q_i \\
\frac{\partial^2\ell}{f_0^2} &=
-\frac{1}{f_0^2} \sum_{i=1}^n x_i \\
\frac{\partial^2\ell}{\beta^2} &= 
-f_1 \sum_{i=1}^n t_i q_i - x_i f_0 q_i (1-q_i)/\lambda_i^2 \\
\frac{\partial^2\ell}{\partial f_0 \partial\beta} &= 
-\frac{f_1}{f_0} \sum_{i=1}^n t_i q_i.
\end{align}
The expression for $\frac{\partial^2\ell}{\partial f_0 \partial\beta}$
is the more complicated one, but fortuately it simplifies at the MLE,
$\beta = \hat{\beta}$ (at the MLE, the gradient of the log-likelihood
with respect to $\beta$ vanishes):
\begin{equation}
\frac{\partial^2\ell}{\beta^2} = -f_1^2 \sum_{i=1}^n x_i (q_i/\lambda_i)^2.
\end{equation}
Therefore, at the MLE $\beta = \hat{\beta}$, the standard error is
\begin{equation}
\mathrm{se}(\hat{\beta}) = \frac{1}{f_1} \times 
\sqrt{\frac{\bar{a}}{\bar{a} \times \bar{c} - \bar{b}^2}},
\end{equation}
where I've defined
\begin{equation*}
\bar{a} = \sum_{i=1}^n x_i, \qquad
\bar{b} = \sum_{i=1}^n t_i q_i, \qquad
\bar{c} = \sum_{i=1}^n x_i (q_i/\lambda_i)^2.
\end{equation*}
The s.e. and $z$-score for the absolute log-fold change statistic
$\beta^{\mathsf{abs}} = \beta/\log 2$ follows
straightforwardly from this result.

\bibliographystyle{siamplain}
\bibliography{diffcount}

\end{document}

