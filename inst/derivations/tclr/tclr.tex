\documentclass[final]{siamart171218}
\usepackage{amsmath}
\usepackage{amssymb}
\usepackage{bm}

\setlength{\oddsidemargin}{0.65in}
\setlength{\evensidemargin}{0.65in}

\title{The ``term-count log-ratio'' statistic for topic modeling analysis
  of differential gene expression}

\author{Peter Carbonetto\thanks{Dept. of Human Genetics and the Research Computing Center, University of Chicago, Chicago, IL}}

\begin{document}

\maketitle

\section{Differential gene expression}

The ``log-fold change'' statistic is commonly used in microarray and
RNA sequencing experiments to quantify expression changes between two
conditions (e.g., \cite{cui-churchill-2003, quackenbush-2002}). To
motivate the ideas below, I write the log-fold change for gene $j$
and condition $k$ as a ratio of two conditional expectations,
\begin{equation*}
\mathsf{lfc}(j,k) = \log_2 \frac{E[x_j \,|\, \mathrm{condition} = k]}
                                {E[x_j \,|\, \mathrm{condition} \neq k]},
\end{equation*}
where $x_j$ is the measured expression level (e.g., UMI count) of gene
$j$. In experiments where the conditions are inferred --- for example,
by running a machine learning algorithm to cluster the expression
profiles --- this quantity could represent the difference in gene
expression between cells inside and outside a cluster.

Supposing $n_k$ out of a total of $n$ gene expression profiles
(cells) are from condition $k$, then $\mathsf{lfc}(j,k)$ can be
computed as
\begin{equation*}
\mathsf{lfc}(j,k) =
\log_2 \left\{\frac{n_{jk}}{n_j - n_{jk}} \times \frac{n - n_k}{n_k} \right\},
\end{equation*}
where $n_j$ is the total expression of gene $j$ among all expression
profiles, and $n_{jk}$ is the total expression of $j$ among all cells
in condition (or cluster) $k$.
  
\section{Poisson non-negative matrix factorization and the multinomial
  topic model}

{\em Add text here.}

\bibliographystyle{siamplain}
\bibliography{tclr}

\end{document}

